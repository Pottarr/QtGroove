\documentclass[12pt]{report} % You can use 'article' or 'book' class as well

% Set up the page
% \usepackage{multicol}
\usepackage{tabularx}
\usepackage{listings}
\usepackage{titlesec}
\usepackage{tocloft}
\usepackage{hyperref} % For clickable links
\usepackage[a4paper, total={6in, 8in}]{geometry} % Adjust the margins
\usepackage{amsmath, amssymb} % For mathematical symbols
\usepackage{graphicx} % For including images
\usepackage{lipsum} % To generate filler text for sample sections
\usepackage{fancyhdr} % For custom headers and footers

\setlength{\parindent}{1cm}  % Indentation of paragraphs (1 cm)
\setlength{\parskip}{5pt}    % No extra space between paragraphs

% Adjust spacing before and after chapter titles
\titlespacing*{\chapter}{0pt}{0.5cm}{0.5cm}

% Adjust spacing before and after section titles
\titlespacing*{\section}{0pt}{0.25cm}{0.25cm}

\setlength{\cftbeforetoctitleskip}{1cm} % Adjust spacing before the TOC title
\setlength{\cftaftertoctitleskip}{1cm}  % Adjust spacing after the TOC title


% Title page setup
% \title{RUSH Document}
% \author{67011090 Chanunyu Chinnawuth \\ 67011352 Theepakorn Phayonrat}

% Begin document
\begin{document}

% Title page
\begin{titlepage}
	\centering
	\vspace*{1cm} % Adjusts vertical space for the image
	% Insert your image (use the actual path and filename of your image)
	\includegraphics[width=0.3\textwidth]{images/KMITL Logo.png} % Adjust width as needed

	\vspace{1cm} % Vertical space after the image
	{\LARGE \textbf{QtGroove Documentation}} \\[0.5cm] % Titl Title
	\vspace{0.5cm}
	{\large \textbf{Object Oriented Programming}} \\[0.5cm]
	{\large \textbf{Software Engineering Program,}} \\[0.5cm]
	{\large \textbf{Department of Computer Engineering,}} \\[0.5cm]
	{\large \textbf{School of Engineering, KMITL}} \\[1cm]
	{\Large 67011090 Chanunyu Chinnawuth \\ 67011352 Theepakorn Phayonrat} \\[0.5cm] % Author
\end{titlepage}

% Preface page
\chapter*{Preface}
\hspace{1cm}This project, QtGroove, was undertaken as part of Object Oriented Programming course in
Software Engineering at KMITL. Throughout this project, We have gained valuable insights into developing
Qt C++ based application, teamwork management and developing workflow. This preface serves to outline
the journey that led to the final outcome, which aims to contribute to the field of Software Engineering
by providing and demonstrating the power of C++ in building efficience multi-platform graphical user interface
(GUI) application.
\newpage

% Abstract page
\chapter*{Abstract (Waiting for finalization)}

\hspace{1cm}This project, titled RUSH: Rust Shell Terminal, presents the design and
implementation of a command-line shell application developed in the Rust programming
language. As part of the Elementary System Programming course in Software Engineering
at KMITL, RUSH was created to develop a fast, and efficient custom shell using the
Rust programming language that provide users with typical features found in standard
command shells (such as Bash, Zsh, etc.), including executing programs, navigating
directories, file management, environment management while utilizing Rust’s safety
guarantees to minimize common security vulnerabilities.


The project demonstrates a range of system programming concepts, including process
management, command parsing, and file handling, showcasing the capabilities of Rust in
building performant and secure applications. RUSH supports essential command-line
functionalities such as a help system, history logging, and automatic configuration of
necessary files, providing a user-friendly and highly customizable terminal experience.

This work aims to contribute to the field of Software Engineering by offering a flexible
and safe alternative to traditional shell interfaces, especially suited for developers
seeking to understand system programming through the Rust language. The final outcome
highlights Rust’s effectiveness in system software development, encouraging further
exploration of Rust for similar high-performance applications.

\newpage
% Table of contents
\tableofcontents
\newpage

% Chapter 1: Introduction
\chapter{Introduction}
\section{Project Overview}


\hspace{1cm}QtGroove is a graphic-based music player written in C++ using the Qt framework. The project
aims to be a lightweight music player with a friendly user interface. 	

QtGroove will have the functions of a typical music player like a file browser, the ability to make
playlists, showing music file info, and having a bit of extra functions like speed up playback or player
customization.


\section{Background}
\hspace{1cm}We wanted to created our own multi-platform GUI music player, which is efficience to navigate
through the UI with low learning curve.

\section{Objective}
\hspace{1cm}This project aims to create a lightweight and multi-platform music player as an alternative
to other music players. The app can great for listening to local music files. The making of this app
also serves as an experience for us to learn C++ and work with the qt framework.

Since this is a duo project, it is a great opportunity to learn teamwork and strive to make the best
products.
 
\chapter{Project Overview}
\section{Design}
\subsection{asdf}
\hspace{1cm}asdf
\subsection{jkl;}
\hspace{1cm}jkl;
\subsection{qwerty}
\hspace{1cm}qwerty

% \begin{table}[ht]
%     \centering
%     \begin{tabularx}{\textwidth}{|X|X|X|}
%         \hline
%         \multicolumn{1}{|c|}{Command} & \multicolumn{1}{|c|}{Description} & \multicolumn{1}{|c|}{Usage} \\
%         \hline
%         \ttfamily{cd} & Change directory. & \ttfamily{cd <path>} \\
%         \hline
%         \ttfamily{clr} & Clear the terminal. & \ttfamily{clr} \\ 
%         \hline
%         \ttfamily{exit} & Exit the terminal & \ttfamily{exit} \\
%         \hline
%         \ttfamily{find} & Find phrase in a file. & \ttfamily{find <phrase> <file>} \\
%         \hline
%         \ttfamily{help} & Display help. & \ttfamily{help <command>} \\
%         \hline
%         \ttfamily{log} & Show command history. & \ttfamily{log} \\
%         \hline
%         \ttfamily{ls} & List files and directories. & \ttfamily{ls} \\
%         \hline
%         \ttfamily{meow} & Concatenate file and print \newline to standard output & \ttfamily{meow <file1> <file2> \newline <file3>...} \\
%         \hline
%         \ttfamily{mkdir} & Create a directory. & \ttfamily{mkdir <directory>} \\ 
%         \hline
%         \ttfamily{mkfile} & Create a file. & \ttfamily{mkfile <file>} \\
%         \hline
%         \ttfamily{shout} & Print a message. & \ttfamily{shout <message>} \\
%         \hline
%     \end{tabularx}
% \end{table}

\newpage

\section{Database}

\hspace{1cm}DB

\subsection{playlist.db}

\hspace{1cm}playlist.db

% \begin{enumerate}
%     \item $|$ (Piping): Insert the returned value from the previous command to the next command.
%     \item Redirecting
%     \begin{enumerate}
%         \item $>$ :  Sends the command’s output to a file, overwriting it if the file exists.
%         \item $>>$ : Sends the output to a file, appending to it if the file exists.
%     \end{enumerate}
% \end{enumerate}

% \subsection{History Log}

% \hspace{1cm}RUSH provides a built-in history log system which stores recently executed commands.
% The history log file (rush.log) is located in:

% \vspace{0.5cm}

% For Windows: \verb|OS_DRIVE:\Users\USERNAME\.rush\rush.log|

% For Unix Based OS: \verb|~/.rush/rush.log|


% Chapter 3: Installation and Execution Guide
\chapter{Installation and Execution Guide}
\section{Git Clone from the Remote Repository}
\begin{lstlisting}[language=Bash ,basicstyle=\footnotesize\ttfamily]
git clone https://github.com/Pottarr/QtGroove.git
cd QtGroove
\end{lstlisting}

% \lipsum[5] % Placeholder text for literature review section

\section{Build and Run the program (Waiting for finalization)}
\begin{lstlisting}[language=Bash ,basicstyle=\footnotesize\ttfamily]
cargo build
cargo run
\end{lstlisting}
% \lipsum[6] % Placeholder text for related studies

% Chapter 4: Summary
\chapter{Summary}
\section{Learning Outcomes}
\begin{itemize}
    \item We have learnt fundamental of concepts of creating good UX and UI.
    \item We have learnt how to develop multi-platform application using C++ Qt.
    \item We have learnt the workflow of project developing.
    \item We have learnt how to use Version Control to help developing application.
\end{itemize}

\section{Accomplishment}
\hspace{1cm}We have created a user friendly multi-platform music player application.

\newpage

% References/Bibliography section
\chapter{References}
% Include your references here in the proper LaTeX format, or use a BibTeX file
\begin{itemize}
    \item Qt Group. (2025). \textit{Qt Documentation}. Retrieved from \url{https://doc.qt.io/}
\end{itemize}

\newpage

\chapter{Appendix}

\section{Demonstration Video}
% \url{https://www.canva.com/design/DAGVhiWupX4/L_cscDEmChqqI4FdjnQ26Q/view?utm_content=DAGVhiWupX4&utm_campaign=designshare&utm_medium=link&utm_source=editor}
% \newpage

\end{document}
